\documentclass[a4paper,12pt]{article}
\usepackage{amsmath}
\usepackage{amssymb}
%\usepackage{amsthm}
%\usepackage{color}
%\usepackage{graphicx}
\usepackage{geometry}
\usepackage{hyperref}
%\usepackage{marginnote}
%\usepackage{mathrsfs}
%\usepackage{syntonly}
%\usepackage{textcomp}
\usepackage{ulem}
\usepackage{verbatim}
%\syntaxonly
%\geometry{a5paper}
\hyphenation{}
\normalem
\hypersetup{
    colorlinks,
    linkcolor=blue,
    filecolor=pink,
    urlcolor=cyan,
    citecolor=red,
}
\title{bitround Manual}
\author{GasinAn}
\begin{document}

    \maketitle

    \tableofcontents

    \section{Overview}

    The input of ``bit round'' is $r$ and $d$, where $d\in[0,\infty)$. For $r\in(0,\infty)$, The output $f(r,d)=\max\{\bar{r}:\bar{r}=mp\leq r+p/2,m\in\mathbb{N}\}$, where $p=\max\{\tilde{p}:\tilde{p}=2^n\leq d,n\in\mathbb{N}\}$. For $r\in(0,\infty)$, The output $f(r,d)=-f(-r,d)$. For $r$ in other situations (including $r$ is $\pm\infty$ or $\text{NaN}$), $f(r,d)=r$. The implement of ``bit round'' may have some undefined behaviors (see Section \ref{sec:implement}).

    \section{The Format of float64}

    Let us define $r(s,e,m)=(-1)^s\times2^e\times m$. The format of float64 has three parts: 1 bit $S$, 11 bits $E$, and 52 bits $T$.
    \begin{itemize}
        \item If $E=0$ and $T=0$, the value is $(-1)^S\times(+0)$.
        \item If $E=0$ and $T\neq0$, the value is $r(S,-1022,0.T)$.
        \item If $1\leq E\leq2046$, the value is $r(S,E-1023,1.T)$.
        \item If $E=2047$ and $T=0$, the value is $(-1)^S\times(+\infty)$.
        \item If $E=2047$ and $T\neq0$, the value is $\text{sNaN}$ when the 1st bit of $T$ is $0$, and $\text{qNaN}$ when the 1st bit of $T$ is $1$.
    \end{itemize}

    \section{Behaviors of Shift Operators in C}

    \verb|a << b| causes the bits in \verb|a| to be shifted to the left by the number of positions specified by \verb|b|, and the bit positions that have been vacated by the shift operation are filled by \verb|0|.

    \verb|a >> b| causes the bits in \verb|a| to be shifted to the right by the number of positions specified by \verb|b|. For unsigned numbers, the bit positions that have been vacated by the shift operation are filled by \verb|0|. For signed numbers, the sign bit is used to fill the vacated bit positions. In other words, if \verb|a| is positive, \verb|0| is used, and if \verb|a| is negative, \verb|1| is used.

    In both \verb|a << b| and \verb|a >> b|, \verb|b| should be non-negative and less than the number of bits in \verb|a|.

    \section{Implement}\label{sec:implement}

    We simply write the format of any float64 number $v$ as $v_1\dots v_{64}$, and $S$, $E$, $T$ of $v$ as $S_v$, $E_v$, $T_v$. We first consider the situation when $1\leq E_r\leq2046$ and $1\leq E_d\leq2046$.

    \subsection{Relationship between $p$ and $d$}

    Since $\tilde{p}=2^n$, $T_p=0\dots0$, and obviously $E_p=E_d$.

    \subsection{When $E_d-E_r>1$}

    When $E_d-E_r>1$, $E_d>E_r+1$, $p/2>\left\lvert r\right\rvert$, $p>\left\lvert r\right\rvert+p/2$, therefore $f(r,d)=0$.

    \subsection{When $E_d-E_r=1$}

    When $E_d-E_r=1$, $E_d=E_r+1$, $p>\left\lvert r\right\rvert\geq p/2$, therefore $f(r,d)=p$.

    \subsection{When $E_d-E_r<1$}

    When $E_d-E_r<1$, We calculate $1.T_r+0.1/2^{E_r-E_d}$, and then make $T_{E_r-E_d}\dots=0$. $1.T_r+0.1/2^{E_r-E_d}$ may make $T$ OVERFLOW, but in this situation, $E$ will become $E+1$ and $T$ will become $0$, which happens to be what we want.

\end{document}
