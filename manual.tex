\documentclass[a4paper,12pt]{article}
\usepackage{amsmath}
%\usepackage{amssymb}
%\usepackage{amsthm}
%\usepackage{color}
%\usepackage{graphicx}
\usepackage{geometry}
\usepackage{hyperref}
%\usepackage{marginnote}
%\usepackage{mathrsfs}
%\usepackage{syntonly}
%\usepackage{textcomp}
\usepackage{ulem}
%\usepackage{verbatim}
%\syntaxonly
%\geometry{a5paper}
\hyphenation{}
\normalem
\hypersetup{
    colorlinks,
    linkcolor=blue,
    filecolor=pink,
    urlcolor=cyan,
    citecolor=red,
}
\title{bitround Manual}
\author{GasinAn}
\begin{document}

    \maketitle

    \tableofcontents

    \section{The Format of float64}

    Let us define $r(s,e,m)=(-1)^s\times2^e\times m$. The format of float64 has three parts: 1 bit $S$, 11 bits $E$, and 52 bits $T$.
    \begin{itemize}
        \item If $E=0$ and $T=0$, the value is $(-1)^S\times(+0)$.
        \item If $E=0$ and $T\neq0$, the value is $r(S,-1022,0.T)$.
        \item If $1\leq E\leq2046$, the value is $r(S,E-1023,1.T)$.
        \item If $E=2047$ and $T=0$, the value is $(-1)^S\times(+\infty)$.
        \item If $E=2047$ and $T\neq0$, the value is $\text{sNaN}$ when the 1st bit of $T$ is $0$, and $\text{qNaN}$ when the 1st bit of $T$ is $1$.
    \end{itemize}

\end{document}
