\documentclass[a4paper,12pt]{article}
\usepackage{amsmath}
\usepackage{amssymb}
%\usepackage{amsthm}
%\usepackage{color}
%\usepackage{graphicx}
\usepackage{geometry}
\usepackage{hyperref}
%\usepackage{marginnote}
%\usepackage{mathrsfs}
%\usepackage{syntonly}
%\usepackage{textcomp}
\usepackage{ulem}
\usepackage{verbatim}
%\syntaxonly
%\geometry{a5paper}
\hyphenation{}
\normalem
\hypersetup{
    colorlinks,
    linkcolor=blue,
    filecolor=pink,
    urlcolor=cyan,
    citecolor=red,
}
\title{bitround Manual}
\author{GasinAn}
\begin{document}

    \maketitle

    \tableofcontents

    \section{Overview}

    The input of ``bit round'' is $r$ and $d$, where $d\in(0,\infty)$. For $r\in(0,\infty)$, The output $f(r,d)=\max\{\bar{r}:\bar{r}\leq r+p/2\}$, where $p=\max\{\tilde{p}:\tilde{p}=2^n\leq d,n\in\mathbb{N}\}$. For $r\in(0,\infty)$, The output $f(r,d)=-f(-r,d)$. For $r$ in other situations (including $r$ is $\pm\infty$ or $\text{NaN}$), $f(r,d)=r$.

    \section{The Format of float64}

    Let us define $r(s,e,m)=(-1)^s\times2^e\times m$. The format of float64 has three parts: 1 bit $S$, 11 bits $E$, and 52 bits $T$.
    \begin{itemize}
        \item If $E=0$ and $T=0$, the value is $(-1)^S\times(+0)$.
        \item If $E=0$ and $T\neq0$, the value is $r(S,-1022,0.T)$.
        \item If $1\leq E\leq2046$, the value is $r(S,E-1023,1.T)$.
        \item If $E=2047$ and $T=0$, the value is $(-1)^S\times(+\infty)$.
        \item If $E=2047$ and $T\neq0$, the value is $\text{sNaN}$ when the 1st bit of $T$ is $0$, and $\text{qNaN}$ when the 1st bit of $T$ is $1$.
    \end{itemize}

    \section{Behaviors of Shift Operators in C}

    \verb|a << b| causes the bits in \verb|a| to be shifted to the left by the number of positions specified by \verb|b|, and the bit positions that have been vacated by the shift operation are filled by \verb|0|.

    \verb|a >> b| causes the bits in \verb|a| to be shifted to the right by the number of positions specified by \verb|b|. For unsigned numbers, the bit positions that have been vacated by the shift operation are filled by \verb|0|. For signed numbers, the sign bit is used to fill the vacated bit positions. In other words, if \verb|a| is positive, \verb|0| is used, and if \verb|a| is negative, \verb|1| is used.

    In both \verb|a << b| and \verb|a >> b|, \verb|b| should be non-negative and less than the number of bits in \verb|a|.

\end{document}
